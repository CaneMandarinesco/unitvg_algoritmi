\documentclass{article}

\usepackage{listings}
\usepackage{amsmath}

\title{Soluzioni per il Problem Set 1}
\date{04-12-2024}
\author{Davide Lights, Ginevra Bru, Alexandra asdf}

\begin{document}

\maketitle
\pagenumbering{gobble}
\newpage
\pagenumbering{arabic}

\section{ Problema 1 }

\subsection{ Codice }

\begin{lstlisting}[language=Python]
algoritmo bilancia(seq S):
    Sia n il numero di elementi in S

    h = 0
    k = 0
    for i = 0 to n do:
        if S[n] == '(' then
            h+=1
        else
            if h > 0 then 
                h-=1
            else k+=1
    
    if h == k then
        return h
    else 
        return inf
\end{lstlisting}

\subsection{ Tempo di Esecuzione }
    Il codice viene eseguito in tempo $\Theta(n)$, dato che itero su ogni carattere della sequenza. 
    La memoria ausiliaria usata e' costante, non vado a richiedere memoria all'interno del ciclo `for`.

\subsection{ Correttezza }
\subsubsection { Lemma }
    Una sequenza $S$, e' $k \text{-bilanciabile}$ se e' solo se 
    ha lo stesso numero $h$ di parentesi aperte (che devono essere chiuse) e $k$ di parentesi chiuse (che devono essere aperte). 

\subparagraph{ Dimostrazione }
    Per assurdo: $k \neq h \to \text{ la sequenza S e' k-bilanciabile}$. Provo a bilanciare la $S$ con $k$. Indico con $k'$ e $h'$ le parentesi aperte e chiuse (da chiudere e aprire)  dopo che ho bilanciato la sequenza. Posso ricavare $k'$ e $h'$ cosi:
    \begin{itemize}
        \item $k' = k - k = 0$, quindi ho chiuso tutte le parentesi aperte!
        \item $h' = h - k > 0$, quindi ho almeno una parentesi chiusa che deve essere aperta, la dimostrazione termina qui.
    \end{itemize}

    Al contrario, per assurdo: $\text{la sequenza S e' k-bilanciabile} \to k \neq h$. 
    Indico con $k'$ e $h'$ le parentesi aperte e chiuse (da chiudere e aprire) dopo che ho bilanciato la sequenza. 
    Dato che $S$ e' $\text{k-bilanciabile}$ mi aspetto che $k'=h'=0$. 
    Come prima mi posso ricavare $k'$ e $h'$:
    \begin{itemize}
        \item $k' = 0 = k-k$
        \item $h' = 0 = h-k$, ma $h\neq k$ quindi deve necessariamente essere che $k==h$
    \end{itemize}

    La dimostrazione funziona anche se provo a bilanciare con $h$ parentesi.
\subsubsection {Conclusione }
    La correttezza dell'algoritmo viene dal Lemma che viene applicato nel codice

\newpage

\section{ Problema 2 }

\subsection{ Codice }
\begin{lstlisting}[language=Python]

algoritmo find_delta(d, t, M, j):
    Sia n il numero di elementi nell array.

    # -- CASO BASE --
    if n-j == 1:
        d1 = M - t[n-1] - 1
        if d1 < d:
            return -1
        return d1

    # -- CASO RICORSIVO --
    d1 = find_delta(d, t, M, j+1)

    if t[j] + d1 > t[j+1]:
        d1 = (M - t[j] - 1) // (n-j)
        if d1 < d:
            return -1
    
    return d1
    

# metodo wrapper per find_delta
procedura alg_find_delta(d, t, M):
    return find_delta(d, t, M, 0)
    
\end{lstlisting}

\subsection{ Tempo di Esecuzione }
E' stato utilizzato un metodo ricorsivo, la ricorsione e' del tipo: 
$$
T(k) = \begin{cases} 
O(1) \;\;\;\;\;\;\;\;\;\;\;\;\;\;\;\;\;\;\;\;\; \text{ se } k = n-1 \\
T(k+1) + O(1) \;\;\; \text{ altrimenti }
\end{cases}
$$
per $0\leq k< n$, segue che l'algoritmo viene eseguito in tempo $\Theta(n)$, verifichiamo che sia un tempo accettabile secondo l'upper-bound $o(n \cdot M)$, usando la definizione di o-piccolo.

Per definizione $M\geq n \cdot\Delta$, quindi $M$ cresce almeno come $n$ (e quindi $M = \Omega(n)$).  
Ipotizzando che $M$ cresca esattamente come $n$ ($M \sim n$) allora, dalla definizione di o-piccolo:
$$
\lim_{ n \to \infty } \frac{n}{n \cdot M} = \frac{1}{M} \sim \frac{1}{n} = 0
$$
Nel caso in cui $M$ cresca ancora più velocemente di $n$, per esempio $n^2, n^3, \dots$ il limite tende sempre a 0, tenendo presente che $M$ non e' mai $< \Delta \cdot n$.

\subsection{ Correttezza }
Nel codice, i tempi di esecuzione si trovano in un array dove l'indice $0$ corrisponde al primo cliente, e l'indice $n-1$ corrisponde all'ultimo cliente.
Possiamo dimostrare la correttezza dell'algoritmo analizzando il caso base e il caso induttivo.
\subsubsection{Caso Base}
Per $k=n-1$, ricado nel caso base, calcolo il valore massimo di $\Delta'$ per servire un solo cliente (l'ultimo), questo valore e' $\Delta' = M - t_{\text{n-1}} - 1$. 
\subsubsection{Caso Induttivo}
Ipotizzando che $\Delta'_{k+1}$ sia il valore massimo trovato al passo $k+1$, devo vedere se questo valore va bene anche al passo $k$ o se devo cambiarlo, ho due scenari possibili.
\subparagraph{Scenario 1}
$t_{k} + \Delta'_{k+1} > t_{k+1}$: questo vuol dire che se servo il cliente $k$ con tempo $\Delta'_{k+1}$, il cliente $k+1$ verrà messo in coda. Se servo i clienti da $k$ a $n$ con il tempo $\Delta'_{k+1}$ andrò sicuramente oltre il tempo limite $M$, il tempo in cui "inizio a servire" almeno uno dei clienti dopo $k$ e' cambiato, devo trovare un nuovo $\Delta'$. Similmente a come faccio per il caso base, il delta massimo possibile e': $\Delta'_{k} = (M-t_{k} -1) / (n-k)$, dove $\Delta'_{k} < \Delta'_{k+1}$.
\subparagraph{Scenario 2}
$t_{k} + \Delta'_{k+1} \leq t_{k+1}$: questo vuol dire che se servo il cliente $k$ con tempo $\Delta'_{k+1}$, mi avanza del tempo che potrei sfruttare per non fare nulla. Ma non posso incrementare $\Delta'_{k+1}$ dato che questo e' il delta massimo per servire i clienti da $k+1$ a $n$ entro il tempo limite $M$, quindi: $\Delta'_{k} = \Delta'_{k+1}$.

\end{document}